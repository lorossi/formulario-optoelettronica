\documentclass{article}
\usepackage[utf8]{inputenc}
\usepackage[italian]{babel}
\usepackage{geometry}
\usepackage{amsfonts} 
\usepackage{ccicons}
\usepackage{url}
\usepackage{hyperref}

\geometry{left = 2 cm, right = 2 cm, bottom = 2 cm, top = 2 cm}
\setlength{\parindent}{0em}
\hypersetup{
	colorlinks=true,
	urlcolor=blue
}

\title{Formulario di Optoelettronica}
\author{Lorenzo Rossi - lorenzo14.rossi@mail.polimi.it}
\date{AA 2019/2020}

\begin{document}
\maketitle

\section{Riguardo al formulario}
Quest'opera è distribuita con Licenza Creative Commons - Attribuzione Non commerciale 4.0 Internazionale \ccbynceu  \newline
Questo formulario verrà espanso (ed, eventualmente, corretto) periodicamente fino a fine corso.
Link repository di GitHub: \url{} link diretto \href{}{qua}. \newline 

\section{Richiami di elettromagnetismo}
\begin{itemize}
	\item Velocità pacchetto d'onda (velocità di gruppo) \( v = \frac{\partial \omega}{\partial k} = \frac{c}{N_g} \)
	\item Velocità di fase \( v_f = \frac{\omega}{k} = \frac{c}{n}  \)
	\item Indice di gruppo \( N_g = n - \lambda_0 \frac{dn}{d\lambda_0} \)
	\item Angoli \( \theta_i \rightarrow \) fascio incidente, \( \theta_r \rightarrow \) fascio riflesso, \( \theta_t \rightarrow \) fascio trasmesso    
	\item Leggi di Snell, con \( n_1 > n_2 \)
	\begin{enumerate}
		\item \( \theta_i = \theta_r \)
		\item \( n_1 \sin(\theta_i) = n_2 \sin(\theta_t) \)
		\item Per \( \theta_i > \theta_c \) si ha riflessione interna totale (TIR), \( \theta_c = \arcsin\frac{n_2}{n_1} \)
	\end{enumerate}
	\item Tunneling ottico 
	\begin{enumerate}
		\item Campo evanescente \( \vec{E_t}(y, z, t) \propto \exp{-\alpha_2 y} \exp{j (\omega t - k_{iz} z} \)
		\item Coefficiente di attenuazione \( \alpha_2 = \frac{2 \pi n_2}{\lambda_0} [(\frac{n_1}{n_2})^2 \sin(\theta_i)^2 - 1] ^ {1/2} = \frac{2 \pi n_2}{\lambda_0} (\frac{\sin(\theta_i)^2}{\sin(\theta_c)^2} - 1) ^ {1/2} \)
		\item Se \( \theta_i > \theta_c\), \(\alpha_2 \) aumenta
	\end{enumerate}
	\item Perdita dovuta alla riflessione \(r = \frac{n_1-n_2}{n_1+n_2} \), \( R = r^2 \)
	\item Perdita dovuta alla trasmissione \( t = \frac{2n_1}{n_1 + n_2} \)
	\item Sfasamento
	\begin{itemize}
		\item Dovuto alla riflessione interna \( \phi = 0 \)
		\item Dovuto alla riflessione esterna \( \phi = \pi \)
		\item Dovuto all'attraversamento di un mezzo di lunghezza \( d \) \( \phi = d \frac{2 \pi n}{\lambda_0} \)
		\item Dovuto alla riflessione interna totale (TIR) \( \tan(\frac{1}{2} \Phi_\perp) = \frac{(\sin(\theta_1)^2 - ( \frac{n_1}{n_2})^2}{\cos(\theta_i)} \), \( \tan(\frac{1}{2} \Phi_\parallel + \frac{1}{2} \pi) = (\frac{n_1}{n_2})^2 \tan(\Theta_\perp) \) 
	\end{itemize}
	\item Coerenza
	\begin{itemize}
		\item Spaziale \( l_c = c \cdot \Delta \nu \)
		\item Temporale \( t_c = \frac{1}{\Delta \nu} \)
	\end{itemize}
	\item Interferenza
	\begin{itemize}
		\item Fasci individuali \( \vec{E_1} = \vec{E_{10}} \exp{i (k r_1 - \omega t + \phi_1} \), \( \vec{E_2} = \vec{E_{20}} \exp{i (k r_2 - \omega t + \phi_2} \)
		\item Campo totale \( \vec{E} = \vec{E_1} + \vec{E_2} \)
		\item Modulo quadro \( |\vec{E}| ^ 2 = |\vec{E_1}| ^ 2 + |\vec{E_2}| ^ 2 + 2 \vec{E_1} \times \vec{E_2} \)
		\item Intensità \( I = I_1 + I_2 + 2 \sqrt{I_1 I_2} \cos(\delta) \) con \( \delta = k(r_2 - r_1) + \phi_2 - \phi_1 \)
	\end{itemize}
	\item Interferenza costruttiva \( \delta = 2m\pi\), \(I = 4 I_1 = 4 I_2 \)
	\item Interferenza distruttiva \( \delta = 2(m + 1)\pi\), \(I = 0 \)
	\item Cavità di Fabry-Perot
	\begin{itemize}
		\item Interferenza costruttiva per \( \nu ? m \frac{c}{2L} \), \(L = m \frac{\lambda}{2} \)
		\item Spettro massimo per \( L = m \frac{\lambda}{2} \)
		\item Massima ampiezza \( I_{max} = \frac{I_0}{(1-R)^2} \)
		\item Mezza larghezza a metà altezza \( \frac{I_{max}}{2} = \frac{I_0}{(1- R)^2 + 4 R \sin(k L) ^ 2} \Rightarrow \sin(k L) = \frac{1 - R}{2 \sqrt{R}}\)
		\item Assumento \( R \approx 1 \) si ottiene \( kL \approx \frac{1 - R}{2 \sqrt{R}} = \frac{2L}{c} \pi \nu \Rightarrow \nu \)
		\item Full width half maximum (FWHM) \( \Delta \nu  = \frac{1}{2} \frac{\frac{C}{2L}}{\frac{\pi \sqrt{R}}{1 - r}} \)
		\item Fattore qualità \( Q = \frac{\nu_m}{\Delta \nu} \)
	\end{itemize}	 
\end{itemize}

\section{Guida d'onda}
\begin{itemize}
	\item Angolo caratteristico del modo \( \theta_m = \sqrt{1 - (\frac{n_2}{n_1}) ^ 2} \)
	\item Condizione di guida d'onda \( [\frac{2 \pi n_1 (2 a)}{\lambda} \cos(\theta_m) - \Phi_m = m \pi \)
	\item Componenti del modo
	\begin{enumerate}
		\item Componente viaggiante \( \beta_m = k_1 \sin(\theta_m)\)
		\item Componente stazionaria \( \kappa_m = k_1 \cos(\theta_m) \)
	\end{enumerate}
	\item Numero di modi
	\begin{itemize}
		\item Numero di modi \( m < \frac{2V - \Phi_m}{\pi} \)
		\item V-number \(V = \frac{2 \pi a n_1}{\lambda} \sqrt{1 - (\frac{n_2}{n_1}) ^ 2} \)
		\item Numero totale di modi \( int(\frac{2V}{\pi}) + 1 \)
		\item Propagazione monomodale \( V < \frac{\pi}{2} \)
		\item Lunghezza di cut-off \( \lambda > \lambda_c = 4a \sqrt{n_1^2 - n_2^2} \)
	\end{itemize}
\end{itemize}

\end{document}