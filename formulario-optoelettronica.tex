\documentclass{article}
\usepackage[utf8]{inputenc}
\usepackage[italian]{babel}
\usepackage{geometry}
\usepackage{amsfonts} 
\usepackage{ccicons}
\usepackage{url}
\usepackage{hyperref}
\usepackage{amsmath}
\usepackage{xfrac}
\usepackage{cellspace}
\usepackage{setspace}
\usepackage{lmodern}
\setlength\cellspacetoplimit{4pt}
\setlength\cellspacebottomlimit{4pt}

\geometry{left = 2 cm, right = 2 cm, bottom = 2 cm, top = 2 cm}
\setlength{\parindent}{0em}
\hypersetup{
	colorlinks=true,
	urlcolor=blue,
    colorlinks,
    linkcolor={black},
    citecolor={black}
}

\pagestyle{plain}
\pagenumbering{gobble}

\title{Formulario di Optoelettronica}
\author{Lorenzo Rossi - lorenzo14.rossi@mail.polimi.it}
\date{AA 2020/2021}

\begin{document}
\maketitle

\vspace{18em}

\large
\begin{doublespacing}\hypersetup{
    urlcolor=black,
  }\urlstyle{same}
  \centerline{Email: \href{mailto://lorenzo14.rossi@mail.polimi.it}{lorenzo14.rossi@mail.polimi.it}}
  \centerline{GitHub: \url{https://github.com/lorossi}}

  \vspace{18em}
  \centerline{Quest'opera è distribuita con Licenza Creative Commons Attribuzione}
  \centerline{Non commerciale 4.0 Internazionale \ccbynceu}
  \centerline{Versione aggiornata al 26/06/2021}
\end{doublespacing}
\newpage


\pagenumbering{roman}
\tableofcontents
\clearpage
\pagenumbering{arabic}
\newpage

\section{Riguardo al formulario}
Quest'opera è distribuita con Licenza Creative Commons - Attribuzione Non commerciale 4.0 Internazionale \ccbynceu \newline
Questo formulario verrà espanso (ed, eventualmente, corretto) periodicamente fino a fine corso (o finché non verrà ritenuto completo). \newline
Link repository di GitHub: \url{https://github.com/lorossi/formulario-optoelettronica} \newline
L'ultima versione può essere scaricata direttamente cliccando \href{https://github.com/lorossi/formulario-optoelettronica/raw/master/formulario-optoelettronica.pdf}{su questo link.} \newline
In questo formulario ho cercato prima di tutto di mettere le formule importanti per la risoluzione degli esercizi, preferendole a quelle utili alla comprensione della materia.

\section{Richiami di base}
\begin{itemize}
  \item Angolo solido:
        \begin{itemize}
          \item Assume valori nell'intervallo \( [0, 4 \pi] \)
          \item Elemento infinitesimo \( d \Omega = 2 \pi \sin(\theta) d \theta \)
          \item Integrale \( \displaystyle \Omega = \int_0^{2 \pi} = 2 \pi \left[  1- \cos(\theta) \right] \)
          \item Unità di misura \textit{steradiante, (sr)}
        \end{itemize}
  \item Spettro di luce visibile:
        \vspace{0.5cm}
        \begin{table}[h]
          \centering
          \renewcommand{\arraystretch}{2}
          \begin{tabular}{|c|c|c|}
            \hline
            \textbf{Colore} & \textbf{Lunghezza d'onda \([nm]\)} & \textbf{Frequenza \(Thz\)} \\ \hline
            Viola           & 380 - 450                          & 670 - 790                  \\ \hline
            Blu             & 450 - 485                          & 620 - 670                  \\ \hline
            Ciano           & 485 - 500                          & 600 - 620                  \\ \hline
            Verde           & 500 - 565                          & 530 - 600                  \\ \hline
            Giallo          & 565 - 590                          & 510 - 530                  \\ \hline
            Arancione       & 590 - 625                          & 480 - 510                  \\ \hline
            Rosso           & 625 - 700                          & 400 - 480                  \\ \hline
          \end{tabular}
        \end{table}
\end{itemize}
\newpage

\section{Onde elettromagnetiche e pacchetti d'onda}
\begin{itemize}
\item Velocità di gruppo \( v = \dfrac{\partial \, \omega}{\partial k} = \dfrac{c}{N_g} \)
\item Velocità di fase \( v_f = \dfrac{\omega}{k} = \dfrac{c}{n}  \)
\item Indice di gruppo \( N_g = n - \lambda_0 \dfrac{\partial \, n}{\partial \lambda_0} \)
\item Variazione della lunghezza d'onda \( \Delta \lambda = \dfrac{c}{\nu^2} | \Delta \nu | \)
\end {itemize}

\subsection{Leggi di Snell}
\begin{itemize}
  \item Angoli (rispetto alla normale della superficie):
        \begin{itemize}
          \item Fascio incidente \( \theta_i \)
          \item Fascio riflesso  \( \theta_r \)
          \item Fascio trasmesso \( \theta_t \)
        \end{itemize}
  \item Prima legge \( \theta_i = \theta_r \)
  \item Seconda legge \( n_1 \sin(\theta_i) = n_2 \sin(\theta_t) \)
  \item Total internal reflection \( \theta_c = \arcsin \left( \dfrac{n_2}{n_1} \right) \)
\end{itemize}

\subsection{Riflessione e trasmissione}
\begin{itemize}
  \item Coefficiente di riflessione \( R = \left( \dfrac{n_2 - n_1}{n_2 + n_1} \right) ^ 2 \)
  \item Coefficiente di trasmissione \( T = \left( \dfrac{2 n_2}{n_2 + n_1} \right) ^ 2 \)
\end{itemize}

\subsection{Tunnelling ottico}
\begin{itemize}
  \item Campo evanescente \( \vec{E} \propto e^{-\alpha_2 z} e ^ { i \omega t } \)
  \item Coefficiente di attenuazione \( \alpha = \dfrac{2 \pi n}{\lambda} \sqrt{ \left( \dfrac{n_1}{n_2} \right) ^ 2 \sin(\theta_i) - 1} \)
  \item Penetrazione \( \delta = \dfrac{1}{\alpha} \)
\end{itemize}

\newpage

\subsection{Sfasamento}
\begin{itemize}
  \item Dovuto alla riflessione interna \( \phi = 0 \)
  \item Dovuto alla riflessione esterna \( \phi = \pi \)
  \item Dovuto all'attraversamento di un mezzo \( \Delta \phi = \dfrac{2 \pi n}{\lambda_0} l \)
  \item Della componente riflessa all'interfaccia:
        \begin{itemize}
          \item Coefficiente perpendicolare \( r_\perp = \dfrac{\cos(\theta_i) - \sqrt{\left( \sfrac{n_2}{n_1} \right) ^ 2 - \sin^2(\theta_i)}}{\cos(\theta_i) + \sqrt{\left( \sfrac{n_2}{n_1} \right) ^ 2 - \sin^2(\theta_i)}} \)
          \item Sfasamento perpendicolare \( \Phi_\perp = 2 \arctan \left[ \dfrac{ \sqrt{\sin^2(\theta_i) - \left( \sfrac{n_2}{n_1} \right) ^ 2 } }{ \cos(\theta_i) } \right] \)
          \item Relazione degli sfasamenti \( \tan \left( \dfrac{1}{2} \Phi_\perp + \dfrac{\pi}{2} \right) = \dfrac{1}{n ^ 2} \tan \left( \dfrac{1}{2} \Phi_\perp \right) \)
        \end{itemize}
\end{itemize}

\subsection{Coerenza}
\begin{itemize}
  \item Spaziale \( l_c = c \cdot \Delta \nu \)
  \item Temporale \( t_c = \sfrac{1}{\Delta \nu} \)
\end{itemize}

\subsection{Interferenza}
\begin{itemize}
  \item Campo totale \( \vec{E} = \vec{E}_1 + \vec{E}_2 \)
  \item Modulo quadro \( \displaystyle |\vec{E}| ^ 2 = |\vec{E}_1| ^ 2 + |\vec{E}_2| ^ 2 + 2 \vec{E}_1 \times \vec{E}_2 \)
  \item Intensità \( \displaystyle  I = I_1 + I_2 + 2 \sqrt{I_1 I_2} \cos(\delta) \) con \( \delta = k(r_2 - r_1) + \phi_2 - \phi_1 \) %% controlla
  \item Interferenza costruttiva \( \delta = 2 m \pi \), \(I = 4 I_1 = 4 I_2 \) \textit{in fase}
  \item Interferenza distruttiva \( \delta = 2 (m + 1) \pi \), \(I = 0 \) \textit{in quadratura}
  \item Interferometro di Young:
        \begin{itemize}
          \item Picchi di interferenza costruttiva \( y = \dfrac{L}{S} \lambda m \)
          \item Intensità dei picchi \( I = I_0  \left[ 1 + cos\left( k \dfrac{S}{L} y \right) \right] \)
          \item Figure di interferenza:
                \begin{itemize}
                  \item Massimi \( k = \dfrac{S}{L} y = 2m \pi \)
                  \item Minimi \( k = \dfrac{S}{L} y = 2(m+1) \pi \)
                \end{itemize}
        \end{itemize}
\end{itemize}

\newpage

\subsection{Riflettore di Bragg}
\begin{itemize}
  \item Spessore \( d = \dfrac{\lambda_0}{n \cdot 2} = \dfrac{\lambda}{2} \)
  \item Riflettanza di un riflettore a N strati \( R = \left( \dfrac{n_1 ^ {2N} - \sfrac{n_0}{n_3} \; n_2 ^ {2N}}{n_1 ^ {2N} + \sfrac{n_0}{n_3} \; n_2 ^ {2N}} \right) ^ 2 \)
  \item Larghezza a metà altezza dello spettro \( \Delta \lambda_{\sfrac{1}{2}} = \dfrac{\Delta \lambda}{\lambda_0} = \dfrac{4}{\pi} \arcsin \left( \dfrac{n_1 - n_2}{n_1 + n_2} \right) \)
\end{itemize}

\subsection{Strato antiriflesso}
\begin{itemize}
  \item Spessore \( d = \dfrac{\lambda_0}{n \cdot 4} = \dfrac{\lambda}{4} \)
  \item Indice di rifrazione \( n_2 = \sqrt{n_1 n_3} \)
  \item Riflettività \( \left( \dfrac{n_0 n_1 - n_2 }{n_0 n_1 + n_ 2} \right) ^ 2 \)
\end{itemize}

\newpage

\section {Cavità di Fabry-Perot}
\begin{itemize}
  \item Frequenze ammesse \(  \nu = m \dfrac{c}{2L} \), \( m \) \textit{numero intero positivo, indice del modo}
  \item Lunghezza della cavità \( L = m \dfrac{\lambda}{2} \)
  \item Free spectral range \( \Delta \nu_{\textnormal{FSR}} = \nu_{m} - \nu_{m-1} = \dfrac{c}{2L} \)
  \item Campo elettrico totale \( \vec{E} = \dfrac{A_0}{1-R \cdot e^ {j 2 k l}} \)
  \item Intensità totale \( I = | E | ^ 2 = \dfrac{A_0^2}{(1-R)^2+ 4 R \sin(kL)^2} \)
  \item Massima ampiezza \( I_{max} = \dfrac{I_0}{(1-R)^2} \)
  \item Finezza spettrale \( F = \dfrac{\pi \sqrt{R}}{1-R} \)
  \item Larghezza a metà altezza dello spettro della frequenza \(\Delta \nu_{\sfrac{1}{2}} = \dfrac{\dfrac{C}{2L}}{\dfrac{\pi \sqrt{R}}{1 - r}} \)
  \item Larghezza a metà altezza dello spettro dell'intensità \(\Delta I_{\sfrac{1}{2}} = \sin(k L) = \dfrac{1 - R}{2 \sqrt{R}} \)
  \item Fattore qualità \( Q = \dfrac{\nu_m}{\Delta \nu} = m F \)
\end{itemize}

\newpage

\section{Guida d'onda}
\begin{itemize}
  \item Angolo caratteristico del modo \( \theta_m = \sqrt{1 - \left(\dfrac{n_2}{n_1} \right) ^ 2} \)
  \item Condizione di guida d'onda \( \dfrac{2 \pi n_1 (2 a)}{\lambda} \cos(\theta_m) - \Phi_m = m \pi \)
  \item Componenti del modo
        \begin{itemize}
          \item Componente viaggiante \( \beta_m = k_1 \sin(\theta_m)\)
          \item Componente stazionaria \( \kappa_m = k_1 \cos(\theta_m) \)
        \end{itemize}
  \item Numero di modi
        \begin{itemize}
          \item V-number \(\displaystyle V = \dfrac{2 \pi a n_1}{\lambda} \sqrt{1 - \left(\dfrac{n_2}{n_1}\right) ^ 2} \)
          \item Numero di modi \( m < \dfrac{2V - \Phi_m}{\pi} \)
          \item Numero totale di modi \( int\left(\dfrac{2V}{\pi} \right) + 1 \)
          \item Propagazione   \( V < \dfrac{\pi}{2} \)
          \item Lunghezza di cut-off \( \displaystyle \lambda_c = 4a \sqrt{n_1^2 - n_2^2} \)
        \end{itemize}
  \item Dispersione
        \begin{itemize}
          \item Intermodale \( \dfrac{\Delta \tau}{L} \approx \dfrac{n_1 - n_2}{c} \)
          \item Di materiale \( \dfrac{\Delta \tau}{L} \approx D_m \Delta \lambda \), con \( D_m = \left| -\dfrac{\lambda}{c} \dfrac{\partial ^ 2 \, n}{\partial \lambda ^ 2} \right| \)
          \item Intramodale \( \Delta \omega = \dfrac{2 \pi}{\Delta \tau} \)
        \end{itemize}
\end{itemize}

\newpage

\section{Fibra ottica}

\begin{itemize}
  \item Perdita per accoppiamento led-fibra \( \alpha_{\textnormal{LF}} = \textnormal{NA} ^ 2 \)
\end{itemize}
\subsection{Fibra step index}
\begin{itemize}
  \item Differenza di indice relativa \(  \Delta = \dfrac{n_1 - n_2}{n_1} \)
  \item Numero di modi \(  M \approx \dfrac{V^2}{2} \)
  \item Dispersione
        \begin{itemize}
          \item Intermodale \(  \dfrac{\Delta \tau}{L} \approx \dfrac{n_1 - n_2}{c} = \dfrac{n_1 \Delta}{c} \)
          \item Di materiale \(  \dfrac{\Delta \tau}{L} \approx | D_m | \Delta \lambda \), con \( D_m =  -\dfrac{\lambda}{c} \dfrac{\partial ^ 2 n}{\partial \lambda ^ 2} \)
          \item Cromatica \( \dfrac{\Delta \tau}{L} = | D_m + D_w | \Delta \lambda = | D_{Cr} | \Delta \lambda \)
        \end{itemize}
  \item Apertura numerica (NA)
        \begin{itemize}
          \item NA \( \displaystyle = \sqrt{n_1^2 - n_2^2} \)
          \item Angolo di accettazione massimo \( \alpha = \arcsin \left( \dfrac{\textnormal{NA}}{n_0} \right) \)
          \item V-Number \( V = \dfrac{2 \pi a}{\lambda} \textnormal{NA} \)
          \item Per \( V < 2.405 \) si ha fibra monomodale
        \end{itemize}
\end{itemize}

\subsection{Fibra GRIN}
\begin{itemize}
  \item \( n \sin(\theta) = \textnormal{cost} \) in tutta la sezione di fibra
\end{itemize}

\newpage

\section{Laser}
\begin{itemize}
  \item Guadagno \( g = \dfrac{c^2}{8 \pi \nu^2 c^2 \tau_{sp} \Delta \nu} (N_2 - N_1) \)
  \item Condizione di soglia \( g_{th} = \alpha_t = \alpha_s + \dfrac{1}{2L} \ln\left(\dfrac{1}{R_1 R_2}\right) \), \( \alpha_s \) perdite interne
  \item Guadagno di soglia \( g_th = \dfrac{1}{2L} \ln\left(\dfrac{1}{R_1 R_2}\right)\)
  \item Guadagno del laser (sopra soglia) \( g = \dfrac{c^2}{8 \pi \nu^2 n ^ 2 \tau_{sp} \Delta \nu} (N_2 - N_1) \)
  \item Potenza di uscita \( P_{out} = \dfrac{N_{ph}}{2} \dfrac{c}{n} h \nu (1 - R_1) A \), \( A \) area della superficie del laser
        \begin{itemize}
          \item In funzione del flusso fotonico \( P_{out} = \dfrac{1}{2} \Phi_{ph} A h \nu_0 ( 1 - R_2) \)
        \end{itemize}
  \item Tempo di spegnimento del laser \( \tau_{ph} = \dfrac{n}{c \cdot \alpha_t} \)
\end{itemize}

\subsection{Laser a gas}
\subsubsection{Effetto Doppler}
\begin{itemize}
  \item Periodo apparente \( T ' = T \left( 1 + \dfrac{v_x}{c} \right) \)
  \item Frequenza apparente \( \nu ' \approx \nu \left(1 - \dfrac{v_x}{c} \right) \)
  \item Allargamento Doppler \( \Delta \nu_{\textnormal{FWHM}} = 2 \nu_0 \sqrt{\dfrac{2 K T \log(2)}{m c ^ 2}} \)
  \item Numero di modi in cavità \(m = \dfrac{ \nu_{\textnormal{FWHM}}}{ \nu_{\textnormal{FSR}}} \)
\end{itemize}

\subsection{Laser a stato solido}
\begin{itemize}
  \item Equazione del diodo \( \dfrac{I}{q w l d} = \dfrac{n}{\tau_r} + C \cdot n \cdot N_{\textnormal{ph}}\)
  \item Numero di fotoni \( N_{\textnormal{ph}} = \dfrac{\tau_{\textnormal{ph}}}{q w l d} ( I - I_{th}) \)
  \item Slope efficiency \( \textnormal{SE} = \dfrac{h c^2 \tau_{\textnormal{ph}}}{2 n  \lambda q L} \)
  \item Potenza in uscita \( P_0 = \textnormal{SE} \cdot (I - I_{\textnormal{th}} ) \)
\end{itemize}

\newpage

\section{Semiconduttori}
\begin{itemize}
  \item Legge dell'azione di massa \( n = p = n_i ^ 2 = N_c N_v \exp{\dfrac{E_g}{k T}} \)
  \item Corrente di deriva \( J = q F (n \mu_n + p \mu_p) \)
  \item Corrente di diffusione \( J = q D_n \dfrac{\partial \, n}{\partial x} - q D_p \dfrac{\partial p}{\partial x} \)
\end{itemize}

\section{LED}
\begin{itemize}
  \item Energia massima dei fotoni \( E_{\textnormal{MAX}} = E_G + \dfrac{KT}{2} \)
  \item Legge di Varshni \( E_G (T) = E_G (0) - \dfrac{A T ^2}{B+T} \)
  \item Brillanza \( \Phi_V = P_{\textnormal{out}} \cdot 683 \; \dfrac{lm}{W} \cdot V (\lambda) \)
\end{itemize}

\subsection{Emettitore lambertiano}
\begin{itemize}
  \item Intensità \( I (\theta) = I_0 \cos(\theta) \)
  \item Potenza emessa \( \displaystyle P_0 = \int\limits_{0}^{2 \pi} I_0 \cos(\theta) d\Omega = \int\limits_{0}^{\sfrac{\pi}{2}} 2 \pi \sin(\theta) d\theta \)
\end{itemize}

\subsection{Efficienza}
\begin{itemize}
  \item Quantica interna \( \eta_{\textnormal{iqe}} = \dfrac{\textnormal{tasso ricombinazione radiativa}}{\textnormal{tasso ricombinazione}} = \dfrac{\sfrac{1}{\tau_r}}{\sfrac{1}{\tau_r} + \sfrac{1}{\tau_{nr}}} = \dfrac{\Phi_{ph}}{\sfrac{I}{q}} = \dfrac{\sfrac{P_{\textnormal{in}}}{h \nu}}{\sfrac{I}{q}}\)
  \item Quantica esterna \( \eta_{\textnormal{eqe}} = \dfrac{\sfrac{P_{\textnormal{out}}}{h \nu}}{\sfrac{I}{q}}\)
  \item Di estrazione \( \eta_{\textnormal{ee}} = \sfrac{ \eta_{\textnormal{eqe}}}{\eta_{\textnormal{iqe}}} \leq 1 \)
  \item Di conversione di potenza \( \eta_{\textnormal{pce}} = \dfrac{P_{\textnormal{out}}}{V I} = \eta_{\textnormal{eqe}} \cdot \dfrac{h \nu}{q V}\)
  \item Luminosa \( \eta_{\textnormal{le}} = \dfrac{\Phi_v}{V I} \)
  \item Larghezza spettro di emissione \( \Delta \nu_{\sfrac{1}{2}} = \dfrac{m k T}{h} \), \(m \approx 3 \)
  \item Larghezza spettro di emissione \( \Delta \lambda_{\sfrac{1}{2}} = \dfrac{m k T \lambda_0^2}{h c} \), \(m \approx 3 \)
\end{itemize}

\newpage

\section{Fotodiodi}
\begin{itemize}
  \item Efficienza quantica \( \eta = \dfrac{\sfrac{I_{\textnormal{ph}}}{q}}{\sfrac{P_0}{h \nu}} \)
  \item Responsività \( R = \dfrac{I_{\textnormal{ph}}}{P_0} = \eta \dfrac{q}{h \nu} = \eta \dfrac{q \lambda}{h c} \)
  \item Capacità di svuotamento \( C = \epsilon_{Si} \dfrac{A}{w d} \)
  \item Tensione di built-in \( \Phi_{\textnormal{bi}} = \dfrac{KT}{q} \ln\left( \dfrac{N_A N_D}{n_i^ 2}\right) \)
  \item Lunghezza zona di svuotamento \( w = \sqrt{ \dfrac{2 \, \epsilon_{\textnormal{si}}}{q} (V_R +  \Phi_{\textnormal{bi}}) \left( \dfrac{1}{N_A} + \dfrac{1}{N_D} \right) } \)
\end{itemize}

\subsection{Fotodiodo a valanga - APD}
\begin{itemize}
  \item Coefficienti di assorbimento:
        \begin{itemize}
          \item degli elettroni \( \alpha_e = A_e \cdot \exp\left(\dfrac{B_e}{F}\right) \)
          \item delle lacune \( \alpha_h = A_h \cdot \exp\left(\dfrac{B_h}{F}\right) \)
          \item rapporto \( k = \dfrac{\alpha_e}{\alpha_h} \)
          \item Fattore di guadagno \( M = \dfrac{1 - k}{e^{-(1-k) \alpha_e w} - k } \)
        \end{itemize}
\end{itemize}

\subsection{Diodo fotovoltaico}
\begin{itemize}
  \item Corrente \( I = - I_{\textnormal{ph}} + I_D \left[\exp\left(\dfrac{qV}{kT}\right) - 1 \right] \)
  \item Fill factor \( \textnormal{FF} = \dfrac{V_m \: I_m}{V_{oc} \: I_{cc}} \)
\end{itemize}

\newpage

\section{Domande di teoria dei temi di esame}

\subsection{01/02/2021}
\begin{enumerate}
  \item Si illustri il bilancio tra assorbimento, emissione spontanea ed emissione stimolata. Sulla base di questo bilancio,
        si ricavino le condizioni di funzionamento di un laser.
  \item Si ricavi il teorema di Shockley-Ramo nei fotodiodi.
\end{enumerate}

\subsection{19/02/2021}
\begin{enumerate}
  \item Si discutano gli effetti di dispersione in una fibra ottica, distinguendo i vari contributi alla dispersione.
  \item Si illustrino i meccanismi di generazione e ricombinazione nei semiconduttori, discutendo le rispettive applicazioni in optoelettronica e i requisiti in termini di band gap.
\end{enumerate}

\subsection{04/09/2020}
\begin{enumerate}
  \item Illustrare il principio di funzionamento di un laser ad eterogiunzione.
  \item Illustrare le caratteristiche I/V di un fotodiodo in assenza e presenza di illuminazione (hv > EG). Nel secondo caso descrivere il comportamento del fotodiodo per tensione negativa, tensione nulla e corrente nulla.
\end{enumerate}

\subsection{20/07/2020}
\begin{enumerate}
  \item Si definisca il V-number per una guida d’onda e per una fibra ottica, illustrando come da esso si possa risalire al numero di modi di propagazione.
  \item Si illustri il principio di funzionamento di un laser per il caso a 3 livelli e quello a 4 livelli, fornendo esempi pratici per entrambi.
\end{enumerate}

\subsection{22/06/2020}
\begin{enumerate}
  \item Illustrare i materiali semiconduttori impiegati nei LED, in particolare le considerazioni che portano alla scelta del materiale in relazione all’efficienza e al colore.
  \item Illustrare il principio di funzionamento di un fotodiodo e la metrica usata per valutare la sua efficienza.
\end{enumerate}

\subsection{13/02/2020}
\begin{enumerate}
  \item Descrivere la struttura ed il principio di funzionamento del laser He-Ne, illustrando le principali caratteristiche della radiazione emessa.
  \item Illustrare il principio di funzionamento di un fotodiodo pn mediante la dimostrazione del teorema di Ramo. \textit{Nota: teorema Shockley-Ramo.}
\end{enumerate}

\subsection{31/01/2020}
\begin{enumerate}
  \item Discutere il problema dell’attenuazione nelle fibre ottiche, mostrando come esso limiti la massima lunghezza di
        una fibra in un sistema di comunicazione.
  \item Illustrare i fenomeni di allargamento della riga di guadagno in una sorgente laser.
\end{enumerate}

\subsection{29/07/2020}
\begin{enumerate}
  \item Discutere il principio di funzionamento dei laser a 3 livelli e 4 livelli.
  \item Descrivere la struttura ed il principio di funzionamento di un fotodiodo a valanga, mettendo in luce i vantaggi
        rispetto ad un fotodiodo pin.
\end{enumerate}

\subsection{20/02/2019}
\begin{enumerate}
  \item Si illustrino la struttura e il funzionamento dei LED bianchi.
  \item Illustrare il principio di funzionamento di un fotodiodo pn mediante la dimostrazione del teorema di Ramo. \textit{Nota: teorema Shockley-Ramo.}
\end{enumerate}

\subsection{12/09/2019}
\begin{enumerate}
  \item Descrivere il principio di funzionamento di una guida d’onda planare, illustrando la condizione di propagazione
        guidata e discutendo il numero di modi di propagazione.
  \item Definire l’efficienza luminosa di un LED, discutendo la differenza rispetto ad una sorgente ad incandescenza.
\end{enumerate}

\subsection{28/06/2019}
\begin{enumerate}
  \item Illustrare il fenomeno della dispersione in una fibra ottica, mostrandone l’impatto sul bit rate di un sistema di
        comunicazione.
  \item Illustrare il principio di funzionamento di un diodo laser ad omogiunzione, mettendo in luce i principali svantaggi
        rispetto a strutture a doppia eterogiunzione.
\end{enumerate}

\subsection{04/02/2019}
\begin{enumerate}
  \item Illustrare i meccanismi di allargamento della riga di guadagno ottico nelle sorgenti laser, riferendosi in particolare
        ai laser He-Ne e Nd3+:YAG.
  \item Illustrare i meccanismi di generazione di luce nei semiconduttori a gap diretto e indiretto, mostrando quale delle
        due categorie si presta meglio come sorgente luminosa.
\end{enumerate}

\subsection{23/07/2019}
\begin{enumerate}
  \item Illustrare il principio di funzionamento di una sorgente laser, descrivendo le principali caratteristiche della
        radiazione emessa.
  \item Discutere la scelta dei materiali nelle sorgenti LED, in particolare per quanto riguarda il colore della luce emessa.
\end{enumerate}

\subsection{05/09/2018}
\begin{enumerate}
  \item Discutere il problema dell’attenuazione in una fibra ottica, descrivendone l’impatto sulla scelta della lunghezza
        d’onda di lavoro in un sistema di comunicazione in fibra.
  \item Illustrare il meccanismo fisico di ionizzazione per impatto, mostrando come possa essere sfruttato nella
        rivelazione di fotoni.
\end{enumerate}

\subsection{29/06/2018}
\begin{enumerate}
  \item Discutere il trade-off tra dispersione ed accoppiamento di potenza in una fibra ottica step-index.
  \item Descrivere la struttura e il principio di funzionamento di un diodo laser VCSEL, mettendo in luce i vantaggi
        rispetto ad una struttura ad emissione laterale.
\end{enumerate}

\subsection{12/02/2018}
\begin{enumerate}
  \item Illustrare i vantaggi di una struttura a quantum well nella generazione di luce coerente e non coerente.
  \item Illustrare il principio di funzionamento di una cella fotovoltaica e discutere l’effetto delle resistenze parassite sulla
        curva caratteristica I-V.
\end{enumerate}

\subsection{27/06/2017}
\begin{enumerate}
  \item Illustrare i fattori che limitano l’efficienza di conversione nelle celle solari.
  \item Illustrare i meccanismi di generazione di luce nei semiconduttori a gap diretto e indiretto, mostrando quale delle
        due categorie si presta meglio come sorgente luminosa.
\end{enumerate}

\subsection{20/07/2017}
\begin{enumerate}
  \item Illustrare i fattori che limitano la massima lunghezza di una fibra ottica in un sistema di comunicazione,
        evidenziando l’origine fisica dei vari contributi e le possibili soluzioni.
  \item Descrivere le caratteristiche che distinguono un diodo LED da un diodo LASER in termini di principio di
        funzionamento, materiali e struttura del dispositivo.
\end{enumerate}

\subsection{}
\begin{enumerate}
  \item Descrivere il funzionamento del laser He-Ne: livelli energetici, pompaggio, meccanismo di emissione laser, e
        spettro d’uscita.
  \item Illustrare il principio di funzionamento di un fotorivelatore a giunzione pn, evidenziando i principali limiti e
        spiegando come essi vengono risolti dal fotodiodo pin.
\end{enumerate}

\iffalse
  \subsection{}
  \begin{enumerate}
    \item
    \item
  \end{enumerate}
\fi

\end{document}