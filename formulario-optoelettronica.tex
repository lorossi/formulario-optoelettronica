\documentclass{article}
\usepackage[utf8]{inputenc}
\usepackage[italian]{babel}
\usepackage{geometry}
\usepackage{amsfonts} 
\usepackage{ccicons}
\usepackage{url}
\usepackage{hyperref}
\usepackage{amsmath}
\usepackage{xfrac}
\usepackage{cellspace}
\usepackage{setspace}
\usepackage{lmodern}
\setlength\cellspacetoplimit{4pt}
\setlength\cellspacebottomlimit{4pt}

\geometry{left = 2 cm, right = 2 cm, bottom = 2 cm, top = 2 cm}
\setlength{\parindent}{0em}
\hypersetup{
	colorlinks=true,
	urlcolor=blue
}

\pagestyle{plain}
\pagenumbering{gobble}

\title{Formulario di Optoelettronica}
\author{Lorenzo Rossi - lorenzo14.rossi@mail.polimi.it}
\date{AA 2019/2020}

\begin{document}
\maketitle

\vspace{18em}

\large
\begin{doublespacing}\hypersetup{
    urlcolor=black,
  }\urlstyle{same}
  \centerline{Email: \href{mailto://lorenzo14.rossi@mail.polimi.it}{lorenzo14.rossi@mail.polimi.it}}
  \centerline{GitHub: \url{https://github.com/lorossi}}

  \vspace{18em}
  \centerline{Quest'opera è distribuita con Licenza Creative Commons Attribuzione}
  \centerline{Non commerciale 4.0 Internazionale \ccbynceu}
  \centerline{Versione aggiornata al 30/04/2021}
\end{doublespacing}
\newpage


\pagenumbering{roman}
\tableofcontents
\clearpage
\pagenumbering{arabic}
\newpage

\section{Riguardo al formulario}
Quest'opera è distribuita con Licenza Creative Commons - Attribuzione Non commerciale 4.0 Internazionale \ccbynceu \newline
Questo formulario verrà espanso (ed, eventualmente, corretto) periodicamente fino a fine corso (o finché non verrà ritenuto completo). \newline
Link repository di GitHub: \url{https://github.com/lorossi/formulario-optoelettronica} \newline
L'ultima versione può essere scaricata direttamente cliccando \href{https://github.com/lorossi/formulario-optoelettronic/raw/master/formulario-optoelettronica.pdf}{su questo link.} \newline
In questo formulario ho cercato prima di tutto di mettere le formule importanti per la risoluzione degli esercizi, preferendole a quelle utili alla comprensione della materia.

\section{Richiami di elettromagnetismo}
\begin{itemize}
  \item Velocità pacchetto d'onda (velocità di gruppo) \( v = \dfrac{\partial \omega}{\partial k} = \dfrac{c}{N_g} \)
  \item Velocità di fase \( v_f = \dfrac{\omega}{k} = \dfrac{c}{n}  \)
  \item Indice di gruppo \( N_g = n - \lambda_0 \dfrac{dn}{d\lambda_0} \)
  \item Angoli \( \theta_i \rightarrow \) fascio incidente, \( \theta_r \rightarrow \) fascio riflesso, \( \theta_t \rightarrow \) fascio trasmesso
  \item Leggi di Snell, con \( n_1 > n_2 \)
        \begin{enumerate}
          \item \( \theta_i = \theta_r \)
          \item \( n_1 \sin(\theta_i) = n_2 \sin(\theta_t) \)
          \item Per \( \theta_i > \theta_c \) si ha riflessione interna totale (TIR), \( \theta_c = \arcsin\dfrac{n_2}{n_1} \)
        \end{enumerate}
  \item Tunneling ottico
        \begin{enumerate}
          \item Campo evanescente \( \vec{E}_t(y, z, t) \propto \exp{-\alpha_2 y} \exp{j (\omega t - k_{iz} z}) \)
          \item Coefficiente di attenuazione \( \alpha_2 = \dfrac{2 \pi n_2}{\lambda_0} \left[\left(\dfrac{n_1}{n_2}\right)^2 \sin(\theta_i)^2 - 1\right] ^ {1/2} = \dfrac{2 \pi n_2}{\lambda_0} \left(\dfrac{\sin(\theta_i)^2}{\sin(\theta_c)^2} - 1 \right) ^ {1/2} \)
          \item Se \( \theta_i > \theta_c\), \(\alpha_2 \) aumenta
        \end{enumerate}
  \item Perdita dovuta alla riflessione \( r = \dfrac{n_1-n_2}{n_1+n_2} \), \( R = r^2 \)
  \item Perdita dovuta alla trasmissione \( t = \dfrac{2n_1}{n_1 + n_2} \)
  \item Sfasamento
        \begin{itemize}
          \item Dovuto alla riflessione interna \( \phi = 0 \)
          \item Dovuto alla riflessione esterna \( \phi = \pi \)
          \item Dovuto all'attraversamento di un mezzo di lunghezza \( d \phi = d \dfrac{2 \pi n}{\lambda_0} \)
          \item Dovuto alla riflessione interna totale (TIR) \( \tan\left(\frac{1}{2} \Phi_\perp\right) = \dfrac{\left(\sin(\theta_1)^2 - ( d\frac{n_1}{n_2})^2 \right)}{\cos(\theta_i)} \), \( \tan(\frac{1}{2} \Phi_\parallel + \frac{1}{2} \pi) = (\dfrac{n_1}{n_2})^2 \tan(\Theta_\perp) \)
        \end{itemize}
  \item Coerenza
        \begin{itemize}
          \item Spaziale \( l_c = c \cdot \Delta \nu \)
          \item Temporale \( t_c = \frac{1}{\Delta \nu} \)
        \end{itemize}
  \item Interferenza
        \begin{itemize}
          \item Fasci individuali \( \vec{E}_1 = \vec{E}_{10} \exp{j (k r_1 - \omega t + \phi_1} \), \( \vec{E_2} = \vec{E_{20}} \exp{j (k r_2 - \omega t + \phi_2}) \)
          \item Campo totale \( \vec{E} = \vec{E}_1 + \vec{E}_2 \)
          \item Modulo quadro \(  |\vec{E}| ^ 2 = |\vec{E}_1| ^ 2 + |\vec{E}_2| ^ 2 + 2 \vec{E}_1 \times \vec{E}_2 \)
          \item Intensità \(  I = I_1 + I_2 + 2 \sqrt{I_1 I_2} \cos(\delta) \) con \( \delta = k(r_2 - r_1) + \phi_2 - \phi_1 \)
        \end{itemize}
  \item Interferenza costruttiva \( \delta = 2m\pi\), \(I = 4 I_1 = 4 I_2 \)
  \item Interferenza distruttiva \( \delta = 2(m + 1)\pi\), \(I = 0 \)
\end{itemize}

\section {Cavità di Fabry-Perot}
\begin{itemize}
  \item Interferenza costruttiva per \(  \nu = m \dfrac{c}{2L} \), \( L = m \dfrac{\lambda}{2} \)
  \item Spettro massimo per \( L = m \dfrac{\lambda}{2} \)
  \item Campo elettrico totale \( \vec{E} = \dfrac{A_0}{1-R \cdot e^ {j 2 k l}} \)
  \item Intensità totale \( I = | E | ^ 2 = \dfrac{A_0^2}{(1-R)^2+ 4 R \sin(kL)^2} \)
  \item Massima ampiezza \( I_{max} = \dfrac{I_0}{(1-R)^2} \)
  \item Mezza larghezza a metà altezza \(  \dfrac{I_{max}}{2} = \dfrac{I_0}{(1- R)^2 + 4 R \sin(k L) ^ 2} \Rightarrow \sin(k L) = \dfrac{1 - R}{2 \sqrt{R}}\)
  \item Finezza spettrale \( F = \dfrac{\pi \sqrt{R}}{1-R} \)
  \item Assumento \( R \approx 1 \) si ottiene \( kL \approx \dfrac{1 - R}{2 \sqrt{R}} = \dfrac{2L}{c} \pi \nu \Rightarrow \nu = \dfrac{1}{2} \dfrac{ \frac{c}{2L}}{\frac{\pi \sqrt{R}}{1-R}}\)
  \item Full width half maximum (FWHM) \( \Delta \nu  = \dfrac{\dfrac{C}{2L}}{\dfrac{\pi \sqrt{R}}{1 - r}} \)
  \item \( \Delta \nu _ {\textnormal{FWHM}} = \dfrac{C}{2 n_s L} \)
  \item Fattore qualità \( Q = \dfrac{\nu_m}{\Delta \nu} = m F \)
\end{itemize}

\section{Guida d'onda}
\begin{itemize}
  \item Angolo caratteristico del modo \( \theta_m = \sqrt{1 - \left(\dfrac{n_2}{n_1} \right) ^ 2} \)
  \item Condizione di guida d'onda \( \dfrac{2 \pi n_1 (2 a)}{\lambda} \cos(\theta_m) - \Phi_m = m \pi \)
  \item Componenti del modo
        \begin{enumerate}
          \item Componente viaggiante \( \beta_m = k_1 \sin(\theta_m)\)
          \item Componente stazionaria \( \kappa_m = k_1 \cos(\theta_m) \)
        \end{enumerate}
  \item Numero di modi
        \begin{itemize}
          \item V-number \(\displaystyle V = \dfrac{2 \pi a n_1}{\lambda} \sqrt{1 - \left(\dfrac{n_2}{n_1}\right) ^ 2} \)
          \item Numero di modi \( m < \dfrac{2V - \Phi_m}{\pi} \)
          \item Numero totale di modi \( int\left(\dfrac{2V}{\pi} \right) + 1 \)
          \item Propagazione monomodale \( V < \dfrac{\pi}{2} \)
          \item Lunghezza di cut-off \( \lambda > \lambda_c = 4a \sqrt{n_1^2 - n_2^2} \)
        \end{itemize}
  \item Dispersione
        \begin{enumerate}
          \item Intermodale
                \begin{itemize}
                  \item Stima della dispersione intermodale \( \Delta \tau = \dfrac{L n_1}{c} - \dfrac{L n_2}{c} \)
                  \item Dispersione per unità di lunghezza \(  \dfrac{\Delta \tau}{L} = \dfrac{n_1 - n_2}{c} \)
                \end{itemize}
          \item Intramodale
                \begin{itemize}
                  \item In presenza di un solo modo (\( \omega < \omega_{cutoff} \)) il pacchetto di distribuisce su un range di frequenze angolari
                  \item \( \Delta \omega = \dfrac{2 \pi}{\Delta \tau} \)
                \end{itemize}
          \item Di materiale
                \begin{itemize}
                  \item Prescinde dalla propagazione in guida e discende dalla dipendenza di n dalla lunghezza d'onda
                  \item \(  v_g = \dfrac{d \omega}{d k} = \dfrac{c}{n - \lambda_0 \dfrac{dn}{d \lambda_0}} = \dfrac{c}{N_g} \)
                \end{itemize}
        \end{enumerate}
\end{itemize}

\section{Bragg reflector}
\begin{itemize}
  \item Massima riflettanza \( R_N = \left(\dfrac{n_1^{2N} - (n_0 / n_3) \cdot n_2^{2N}}{n_1^{2N} + (n_0 / n_3) \cdot n_2^{2N}} \right) ^ 2 \) con \( N \) numero di doppi strati utilizzati
  \item Riflettanza \( \displaystyle R = \sqrt{R_1 R_2} \)
  \item Larghezza di banda \( \Delta \lambda \approx \dfrac{4}{\pi} \arcsin \left( \dfrac{n_1 - n_2}{n_1 + n_2} \right) \)
\end{itemize}

\section{Fibra ottica}
\subsection{Fibra step index}
\begin{itemize}
  \item Differenza di indice relativa \(  \Delta = \dfrac{n_1 - n_2}{n_1} \)
  \item V number \(  V = \dfrac{2 \pi a}{\lambda} \sqrt{n_1^2 - n_2^2} = \dfrac{2 \pi a}{\lambda} \sqrt{2 n n_1 \Delta}\) con \(  n = \dfrac{n_1 + n_2}{2} \). Per \( V < 2.405 \) ho fibra monomodale.
  \item Numero di modi \(  M \approx \dfrac{V^2}{2} \)
  \item Attenuazione in fibra \(  \alpha = - \dfrac{1}{P} \dfrac{dP}{dx} \rightarrow P = P_0 e^{-\alpha L}\), \(   E = E_0 e^{-\alpha L / 2} \)
  \item Dispersione
        \begin{enumerate}
          \item Intermodale \(  \dfrac{\Delta \tau}{L} \approx \dfrac{n_1 - n_2}{c} = \dfrac{n_1 \Delta}{c} \)
          \item Di materiale \(  \dfrac{\Delta \tau}{L} = | D_m | \Delta \lambda \) con \( D_m = -\dfrac{\lambda}{c} \dfrac{d^2 n}{d \lambda^2} \)
          \item Di guida/cromatica \(  \dfrac{\Delta \tau}{L} = | D_w | \Delta \lambda \)
          \item Sommando \( D_m \) e \( D_w \) si ottiene la dispersione cromatica \( \dfrac{\Delta \tau}{L} = | D_m + D_w | \Delta \lambda = | D_{Cr} | \Delta \lambda \)
        \end{enumerate}
  \item Apertura numerica (NA)
        \begin{enumerate}
          \item NA \( \displaystyle = \sqrt{n_1^2 - n_2^2} \)
          \item Angolo di accettanza massimo \( \alpha = \arcsin \left( \dfrac{\textnormal{NA}}{n_0} \right) \)
          \item V-Number \( V = \dfrac{2 \pi a}{\lambda} \textnormal{NA} \)
        \end{enumerate}
\end{itemize}
\end{document}