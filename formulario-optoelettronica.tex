\documentclass{article}
\usepackage[utf8]{inputenc}
\usepackage[italian]{babel}
\usepackage{geometry}
\usepackage{amsfonts} 
\usepackage{ccicons}
\usepackage{url}
\usepackage{hyperref}
\usepackage{amsmath}
\usepackage{xfrac}
\usepackage{cellspace}
\usepackage{setspace}
\usepackage{lmodern}
\setlength\cellspacetoplimit{4pt}
\setlength\cellspacebottomlimit{4pt}

\geometry{left = 2 cm, right = 2 cm, bottom = 2 cm, top = 2 cm}
\setlength{\parindent}{0em}
\hypersetup{
	colorlinks=true,
	urlcolor=blue,
    colorlinks,
    linkcolor={black},
    citecolor={black}
}

\pagestyle{plain}
\pagenumbering{gobble}

\title{Formulario di Optoelettronica}
\author{Lorenzo Rossi - lgorenzo14.rossi@mail.polimi.it}
\date{AA 2020/2021}

\begin{document}
\maketitle

\vspace{18em}

\large
\begin{doublespacing}\hypersetup{
    urlcolor=black,
  }\urlstyle{same}
  \centerline{Email: \href{mailto://lorenzo14.rossi@mail.polimi.it}{lorenzo14.rossi@mail.polimi.it}}
  \centerline{GitHub: \url{https://github.com/lorossi}}

  \vspace{18em}
  \centerline{Quest'opera è distribuita con Licenza Creative Commons Attribuzione}
  \centerline{Non commerciale 4.0 Internazionale \ccbynceu}
  \centerline{Versione aggiornata al 11/06/2021}
\end{doublespacing}
\newpage


\pagenumbering{roman}
\tableofcontents
\clearpage
\pagenumbering{arabic}
\newpage

\section{Riguardo al formulario}
Quest'opera è distribuita con Licenza Creative Commons - Attribuzione Non commerciale 4.0 Internazionale \ccbynceu \newline
Questo formulario verrà espanso (ed, eventualmente, corretto) periodicamente fino a fine corso (o finché non verrà ritenuto completo). \newline
Link repository di GitHub: \url{https://github.com/lorossi/formulario-optoelettronica} \newline
L'ultima versione può essere scaricata direttamente cliccando \href{https://github.com/lorossi/formulario-optoelettronica/raw/master/formulario-optoelettronica.pdf}{su questo link.} \newline
In questo formulario ho cercato prima di tutto di mettere le formule importanti per la risoluzione degli esercizi, preferendole a quelle utili alla comprensione della materia.

\section{Richiami di base}
\begin{itemize}
  \item Angolo solido:
        \begin{itemize}
          \item Assume valori nell'intervallo \( [0, 4 \pi] \)
          \item Elemento infinitesimo \( d \Omega = 2 \pi \sin(\theta) d \theta \)
          \item Integrale \( \displaystyle \Omega = \int_0^{2 \pi} = 2 \pi \left[  1- \cos(\theta) \right] \)
          \item Unità di misura \textit{steradiante}
        \end{itemize}
  \item Spettro di luce visibile:
        \vspace{0.5cm}
        \begin{table}[h]
          \centering
          \renewcommand{\arraystretch}{2}
          \begin{tabular}{|c|c|c|}
            \hline
            \textbf{Colore} & \textbf{Lunghezza d'onda \([nm]\)} & \textbf{Frequenza \(Thz\)} \\ \hline
            Viola           & 380 - 450                          & 670 - 790                  \\ \hline
            Blu             & 450 - 485                          & 620 - 670                  \\ \hline
            Ciano           & 485 - 500                          & 600 - 620                  \\ \hline
            Verde           & 500 - 565                          & 530 - 600                  \\ \hline
            Giallo          & 565 - 590                          & 510 - 530                  \\ \hline
            Arancione       & 590 - 625                          & 480 - 510                  \\ \hline
            Rosso           & 625 - 700                          & 400 - 480                  \\ \hline
          \end{tabular}
        \end{table}
\end{itemize}
\newpage

\section{Onde elettromagnetiche e pacchetti d'onda}
\begin{itemize}
\item Velocità di gruppo \( v = \dfrac{\partial \omega}{\partial k} = \dfrac{c}{N_g} \)
\item Velocità di fase \( v_f = \dfrac{\omega}{k} = \dfrac{c}{n}  \)
\item Indice di gruppo \( N_g = n - \lambda_0 \dfrac{\partial n}{\partial \lambda_0} \)
\item Variazione della lunghezza d'onda \( \Delta \lambda = \dfrac{c}{\nu^2} | \Delta \nu | \)
\end {itemize}

\subsection{Leggi di Snell}
\begin{itemize}
  \item Angoli (rispetto alla normale della superficie):
        \begin{itemize}
          \item Fascio incidente \( \theta_i \)
          \item Fascio riflesso  \( \theta_r \)
          \item Fascio trasmesso \( \theta_t \)
        \end{itemize}
  \item Prima legge \( \theta_i = \theta_r \)
  \item Seconda legge \( n_1 \sin(\theta_i) = n_2 \sin(\theta_t) \)
  \item Total internal reflection \( \theta_c = \arcsin\dfrac{n_2}{n_1} \)
\end{itemize}

\subsection{Riflessione e trasmissione}
\begin{itemize}
  \item Coefficiente di riflessione \( R = \left( \dfrac{n_2 - n_1}{n_2 + n_1} \right) ^ 2 \)
  \item Coefficiente di trasmissione \( T = \left( \dfrac{2 n_2}{n_2 + n_1} \right) ^ 2 \)
\end{itemize}

\subsection{Tunneling ottico}
\begin{itemize}
  \item Campo evanescente \( \vec{E} \propto \exp{\{-\alpha_2 z\}} \exp{\{ i \omega t \}} \)
  \item Coefficiente di attenuazione \( \alpha = \dfrac{2 \pi n}{\lambda} \sqrt{ \left( \dfrac{n_1}{n_2} \right) ^ 2 \sin(\theta_i) - 1} \)
  \item Penetrazione \( \delta = \dfrac{1}{\alpha} \)
\end{itemize}

\newpage

\subsection{Sfasamento}
\begin{itemize}
  \item Dovuto alla riflessione interna \( \phi = 0 \)
  \item Dovuto alla riflessione esterna \( \phi = \pi \)
  \item Dovuto all'attraversamento di un mezzo \( \Delta \phi = \dfrac{2 \pi n}{\lambda_0} l \)
  \item Della componente riflessa all'interfaccia:
        \begin{itemize}
          \item Coefficiente perpendicolare \( r_\perp = \dfrac{\cos(\theta_i) - \sqrt{\left( \sfrac{n_2}{n_1} \right) ^ 2 - \sin^2(\theta_i)}}{\cos(\theta_i) + \sqrt{\left( \sfrac{n_2}{n_1} \right) ^ 2 - \sin^2(\theta_i)}} \)
          \item Sfasamento perpendicolare \( \Phi_\perp = 2 \arctan \left[ \dfrac{ \sqrt{\sin^2(\theta_i) - \left( \sfrac{n_2}{n_1} \right) ^ 2 } }{ \cos(\theta_i) } \right] \)
          \item Relazione degli sfasamenti \( \tan \left( \dfrac{1}{2} \Phi_\perp + \dfrac{\pi}{2} \right) = \dfrac{1}{n ^ 2} \tan \left( \dfrac{1}{2} \Phi_\perp \right) \)
        \end{itemize}
\end{itemize}

\subsection{Coerenza}
\begin{itemize}
  \item Spaziale \( l_c = c \cdot \Delta \nu \)
  \item Temporale \( t_c = \frac{1}{\Delta \nu} \)
\end{itemize}

\subsection{Interferenza}
\begin{itemize}
  \item Campo totale \( \vec{E} = \vec{E}_1 + \vec{E}_2 \)
  \item Modulo quadro \( \displaystyle |\vec{E}| ^ 2 = |\vec{E}_1| ^ 2 + |\vec{E}_2| ^ 2 + 2 \vec{E}_1 \times \vec{E}_2 \)
  \item Intensità \( \displaystyle  I = I_1 + I_2 + 2 \sqrt{I_1 I_2} \cos(\delta) \) con \( \delta = k(r_2 - r_1) + \phi_2 - \phi_1 \) %% controlla
  \item Interferenza costruttiva \( \delta = 2 m \pi \), \(I = 4 I_1 = 4 I_2 \) \textit{in fase}
  \item Interferenza distruttiva \( \delta = 2 (m + 1) \pi \), \(I = 0 \) \textit{in quadratura}
  \item Interferomero di Young:
        \begin{itemize}
          \item Picchi di interferenza costruttiva \( y = \dfrac{L}{S} \lambda m \)
          \item Intensità dei picchi \( I = I_0  \left[ 1 + cos\left( k \dfrac{S}{L} y \right) \right] \)
          \item Figure di interferenza:
                \begin{itemize}
                  \item Massimi \( k = \dfrac{S}{L} y = 2m \pi \)
                  \item Minimi \( k = \dfrac{S}{L} y = 2(m+1) \pi \)
                \end{itemize}
        \end{itemize}
\end{itemize}

\newpage

\subsection{Riflettore di Bragg}
\begin{itemize}
  \item Spessore \( d = \dfrac{\lambda_0}{n \cdot 2} = \dfrac{\lambda}{4} \)
  \item Riflettanza di un riflettore a N strati \( R = \left( \dfrac{n_1 ^ {2N} - \sfrac{n_0}{n_3} \; n_2 ^ {2N}}{n_1 ^ {2N} + \sfrac{n_0}{n_3} \; n_2 ^ {2N}} \right) ^ 2 \)
  \item Larghezza a metà altezza dello spettro delle lunghezze d'onda \( \Delta \lambda_{\sfrac{1}{2}} = \dfrac{\Delta \lambda}{\lambda_0} = \dfrac{4}{\pi} \arcsin \left( \dfrac{n_1 - n_2}{n_1 + n_2} \right) \)
\end{itemize}

\subsection{Strato antiriflesso}
\begin{itemize}
  \item Spessore \( d = \dfrac{\lambda_0}{n \cdot 4} = \dfrac{\lambda}{4} \)
  \item Indice di rifrazione \( n_2 = \sqrt{n_1 n_3} \)
  \item Riflettività \( \left( \dfrac{n_0 n_1 - n_2 }{n_0 n_1 + n_ 2} \right) ^ 2 \)
\end{itemize}

\newpage

\section {Cavità di Fabry-Perot}
\begin{itemize}
  \item Frequenze ammesse \(  \nu = m \dfrac{c}{2L} \), \( m \) \textit{numero intero positivo, indice del modo}
  \item Free spectral range \( \Delta \nu_{\textnormal{FSR}} = \nu_{m} - \nu_{m-1} = \dfrac{c}{2L} \)
  \item Campo elettrico totale \( \vec{E} = \dfrac{A_0}{1-R \cdot e^ {j 2 k l}} \)
  \item Intensità totale \( I = | E | ^ 2 = \dfrac{A_0^2}{(1-R)^2+ 4 R \sin(kL)^2} \)
  \item Massima ampiezza \( I_{max} = \dfrac{I_0}{(1-R)^2} \)
  \item Finezza spettrale \( F = \dfrac{\pi \sqrt{R}}{1-R} \)
  \item Larghezza a metà altezza dello spettro della frequenza \(\Delta \nu_{\sfrac{1}{2}} = \dfrac{\dfrac{C}{2L}}{\dfrac{\pi \sqrt{R}}{1 - r}} \)
  \item Larghezza a metà altezza dello spettro dell'intensità \(\Delta I_{\sfrac{1}{2}} = \sin(k L) = \dfrac{1 - R}{2 \sqrt{R}} \)
  \item Fattore qualità \( Q = \dfrac{\nu_m}{\Delta \nu} = m F \)
\end{itemize}

\newpage

\section{Guida d'onda}
\begin{itemize}
  \item Angolo caratteristico del modo \( \theta_m = \sqrt{1 - \left(\dfrac{n_2}{n_1} \right) ^ 2} \)
  \item Condizione di guida d'onda \( \dfrac{2 \pi n_1 (2 a)}{\lambda} \cos(\theta_m) - \Phi_m = m \pi \)
  \item Componenti del modo
        \begin{itemize}
          \item Componente viaggiante \( \beta_m = k_1 \sin(\theta_m)\)
          \item Componente stazionaria \( \kappa_m = k_1 \cos(\theta_m) \)
        \end{itemize}
  \item Numero di modi
        \begin{itemize}
          \item V-number \(\displaystyle V = \dfrac{2 \pi a n_1}{\lambda} \sqrt{1 - \left(\dfrac{n_2}{n_1}\right) ^ 2} \)
          \item Numero di modi \( m < \dfrac{2V - \Phi_m}{\pi} \)
          \item Numero totale di modi \( int\left(\dfrac{2V}{\pi} \right) + 1 \)
          \item Propagazione monomodale \( V < \dfrac{\pi}{2} \)
          \item Lunghezza di cut-off \( \displaystyle \lambda_c = 4a \sqrt{n_1^2 - n_2^2} \)
        \end{itemize}
  \item Dispersione
        \begin{itemize}
          \item Intermodale \( \dfrac{\Delta \tau}{L} \approx \dfrac{n_1 - n_2}{c} \)
          \item Di materiale \( \dfrac{\Delta \tau}{L} \approx D_m \Delta \lambda \), con \( D_m = \left| -\dfrac{\lambda}{c} \dfrac{\partial ^ 2}{\partial \lambda ^ 2} \right| \)
          \item Intramodale \( \Delta \omega = \dfrac{2 \pi}{\Delta \tau} \)
        \end{itemize}
\end{itemize}

\newpage

\section{Fibra ottica}
\subsection{Fibra step index}
\begin{itemize}
  \item Differenza di indice relativa \(  \Delta = \dfrac{n_1 - n_2}{n_1} \)
  \item Numero di modi \(  M \approx \dfrac{V^2}{2} \)
  \item Dispersione
        \begin{itemize}
          \item Intermodale \(  \dfrac{\Delta \tau}{L} \approx \dfrac{n_1 - n_2}{c} = \dfrac{n_1 \Delta}{c} \)
          \item Di materiale \(  \dfrac{\Delta \tau}{L} \approx D_m  \Delta \lambda \), con \( D_m = \left| -\dfrac{\lambda}{c} \dfrac{\partial ^ 2}{\partial \lambda ^ 2} \right| \)
          \item Cromatica \( \dfrac{\Delta \tau}{L} = | D_m + D_w | \Delta \lambda = | D_{Cr} | \Delta \lambda \)
        \end{itemize}
  \item Apertura numerica (NA)
        \begin{itemize}
          \item NA \( \displaystyle = \sqrt{n_1^2 - n_2^2} \)
          \item Angolo di accettazione massimo \( \alpha = \arcsin \left( \dfrac{\textnormal{NA}}{n_0} \right) \)
          \item V-Number \( V = \dfrac{2 \pi a}{\lambda} \textnormal{NA} \)
          \item Per \( V < 2.405 \) ho fibra monomodale.
        \end{itemize}
\end{itemize}

\subsection{Fibra GRIN}
\begin{itemize}
  \item \( n \sin(\theta) = \textnormal{cost} \) in tutta la sezione di fibra
\end{itemize}

\newpage

\section{Laser}
\begin{itemize}
  \item Guadagno \( g = \dfrac{c^2}{8 \pi \nu^2 c^2 \tau_{sp} \Delta \nu} (N_2 - N_1) \)
  \item Condizione di soglia \( g_{th} = \alpha_t = \alpha_s + \dfrac{1}{2L} \ln\left(\dfrac{1}{R_1 R_2}\right) \), \( \alpha_s \) perdite interne
  \item Guadagno di soglia \( g_th = \dfrac{1}{2L} \ln\left(\dfrac{1}{R_1 R_2}\right)\)
  \item Guadagno del laser (sopra soglia) \( g = \dfrac{c^2}{8 \pi \nu^2 n ^ 2 \tau_{sp} \Delta \nu} (N_2 - N_1) \)
  \item Potenza di uscita \( P_{out} = \dfrac{N_{ph}}{2} \dfrac{c}{n} h \nu (1 - R_1) A \), \( A \) area della superficie del laser
        \begin{itemize}
          \item In funzione del flusso fotonico \( P_{out} = \dfrac{1}{2} \Phi_{ph} A h \nu_0 ( 1 - R_2) \)
        \end{itemize}
  \item Tempo di spegnimento del laser \( \tau_{ph} = \dfrac{n}{c \cdot \alpha_t} \)
\end{itemize}

\subsection{Laser a gas}
\subsubsection{Effetto Doppler}
\begin{itemize}
  \item Periodo apparente \( T ' = T \left( 1 + \dfrac{v_x}{c} \right) \)
  \item Frequenza apparente \( \nu ' \approx \nu \left(1 - \dfrac{v_x}{c} \right) \)
  \item Allargamento Doppler \( \Delta \nu_{\textnormal{FWHM}} = 2 \nu_0 \sqrt{\dfrac{2 K T \log(2)}{m c ^ 2}} \)
  \item Numero di modi in cavità \(m = \dfrac{ \nu_{\textnormal{FWHM}}}{ \nu_{\textnormal{FSR}}} \)
\end{itemize}

\subsection{Laser a stato solido}
\begin{itemize}
  \item Equazione del diodo \( \dfrac{I}{q w l d} = \dfrac{n}{\tau_r} + C \cdot n \cdot N_{\textnormal{ph}}\)
  \item Numero di fotoni \( N_{\textnormal{ph}} = \dfrac{\tau_{\textnormal{ph}}}{q w l d} ( I - I_{th}) \)
  \item Slope efficiency \( \textnormal{SE} = \dfrac{h c^2 \tau_{\textnormal{ph}}}{2 n  \lambda q L} \)
  \item Potenza in uscita \( P_0 = \textnormal{SE} \cdot (I - I_{\textnormal{th}} ) \)
\end{itemize}

\newpage

\section{Semiconduttori}
\begin{itemize}
  \item Legge dell'azione di massa \( n = p = n_i ^ 2 = N_c N_v \exp{\dfrac{E_g}{k T}} \)
  \item Corrente di deriva \( J = q F (n \mu_n + p \mu_p) \)
  \item Corrente di diffusione \( J = q D_n \dfrac{\partial n}{\partial x} - q D_p \dfrac{\partial p}{\partial x} \)
\end{itemize}

\section{LED}
\begin{itemize}
  \item Energia massima dei fotoni \( E_{\textnormal{MAX}} = E_G + \dfrac{KT}{2} \)
  \item Legge di Varshni \( E_G (T) = E_G (0) - \dfrac{A T ^2}{B+T} \)
  \item Brillanza \( \Phi_V = P_{\textnormal{out}} \cdot 683 \; \dfrac{lm}{W} \cdot V (\lambda) \)
\end{itemize}

\subsection{Emettitore lambertiano}
\begin{itemize}
  \item Intensità \( I (\theta) = I_0 \cos(\theta) \)
  \item Potenza emessa \( \displaystyle P_0 = \int\limits_{0}^{2 \pi} I_0 \cos(\theta) d\Omega = \int\limits_{0}^{\sfrac{\pi}{2}} 2 \pi \sin(\theta) d\theta \)
\end{itemize}

\subsection{Efficienza}
\begin{itemize}
  \item Quantica interna \( \eta_{\textnormal{iqe}} = \dfrac{\textnormal{tasso ricombinazione radiativa}}{\textnormal{tasso ricombinazione}} = \dfrac{\sfrac{1}{\tau_r}}{\sfrac{1}{\tau_r} + \sfrac{1}{\tau_{nr}}} = \dfrac{\Phi_{ph}}{\sfrac{I}{q}} = \dfrac{\sfrac{P_{\textnormal{in}}}{h \nu}}{\sfrac{I}{q}}\)
  \item Quantica esterna \( \eta_{\textnormal{eqe}} = \dfrac{\sfrac{P_{\textnormal{out}}}{h \nu}}{\sfrac{I}{q}}\)
  \item Di estrazione \( \eta_{\textnormal{ee}} = \sfrac{ \eta_{\textnormal{eqe}}}{\eta_{\textnormal{iqe}}} \leq 1 \)
  \item Di conversione di potenza \( \eta_{\textnormal{pce}} = \dfrac{P_{\textnormal{out}}}{V I} = \eta_{\textnormal{eqe}} \cdot \dfrac{h \nu}{q V}\)
  \item Luminosa \( \eta_{\textnormal{le}} = \dfrac{\Phi_v}{V I} \)
  \item Larghezza spettro di emissione \( \Delta \nu_{\sfrac{1}{2}} = \dfrac{m k T}{h} \), \(m \approx 3 \)
  \item Larghezza spettro di emissione \( \Delta \lambda_{\sfrac{1}{2}} = \dfrac{m k T \lambda_0^2}{h c} \), \(m \approx 3 \)
\end{itemize}

\newpage

\section{Fotodiodi}
\begin{itemize}
  \item Efficienza quantica \( \eta = \dfrac{\sfrac{I_{\textnormal{ph}}}{q}}{\sfrac{P_0}{h \nu}} \)
  \item Responsività \( R = \dfrac{I_{\textnormal{ph}}}{P_0} = \eta \dfrac{q}{h \nu} = \eta \dfrac{q \lambda}{h c} \)
  \item Capacità di svuotamento \( C = \epsilon_{Si} \dfrac{A}{w d} \)
  \item Tensione di built-in \( \Phi_{\textnormal{bi}} = \dfrac{KT}{q} \ln\left( \dfrac{N_A N_D}{n_i^ 2}\right) \)
  \item Lunghezza zona di svuotamento \( w = \sqrt{ \dfrac{2 \epsilon_{Si}}{q} (V_R +  \Phi_{\textnormal{bi}}) \left( \dfrac{1}{N_A} + \dfrac{1}{N_d} \right) } \)
\end{itemize}

\subsection{Fotodiodo a valanga - APD}
\begin{itemize}
  \item Coefficienti di assorbimento:
        \begin{itemize}
          \item degli elettroni \( \alpha_e = A_e \cdot \exp\left(\dfrac{B_e}{F}\right) \)
          \item delle lacune \( \alpha_h = A_h \cdot \exp\left(\dfrac{B_h}{F}\right) \)
          \item rapporto \( k = \dfrac{\alpha_e}{\alpha_h} \)
          \item Fattore di guadagno \( M = \dfrac{1 - k}{e^{-(1-k) \alpha_e w} - k } \)
        \end{itemize}
  \item Fattore di guadagno \( M = \dfrac{I_{\textnormal{ph}}}{I_{\textnormal{ph}_0}} \approx \left[ 1-\left( \dfrac{V}{V_\textnormal{BD}} \right) ^ m \right] ^ {9} \)
\end{itemize}

\subsection{Diodo fotovoltaico}
\begin{itemize}
  \item Corrente \( I = - I_{\textnormal{ph}} + I_D \left[\exp\left(\dfrac{qV}{kT}\right) - 1 \right] \)
  \item Fill factor \( \textnormal{FF} = \dfrac{V_m \: I_m}{V_{oc} \: I_{cc}} \)
\end{itemize}


\end{document}